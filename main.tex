\include{preamble}
\begin{document}

\title{Axion-active phonons in MnBi$_2$Te$_4$}

\author{Trey Cole}
\affiliation{
Department of Physics \& Astronomy, Center for Materials Theory, Rutgers University, Piscataway, New Jersey 08854, USA}

\author{Daniel Seleznev}
\affiliation{
Department of Physics \& Astronomy, Center for Materials Theory, Rutgers University, Piscataway, New Jersey 08854, USA}
\affiliation{
Department of Physics, University of Texas at Austin, Austin, Texas 78712, USA}

\author{Andrea Urru}
\affiliation{
Department of Physics \& Astronomy, Center for Materials Theory, Rutgers University, Piscataway, New Jersey 08854, USA}


\author{David Vanderbilt}
\affiliation{
Department of Physics \& Astronomy, Center for Materials Theory, Rutgers University, Piscataway, New Jersey 08854, USA}

\begin{abstract}
MnBi$_2$Te$_4$ is an antiferromagnetic topological insulator whose axion angle is quantized by the presence of inversion $\mathcal{P}$ and time-reversal times half-translation $\Theta = \mathcal{T}S_{1/2}$ symmetries \cite{Turner2012}
. In the presence of axion-active phonons that break both of these symmetries simultaneously, we expect that the axion angle will vary away from its nominal value of $\theta=\pi$. In this work, we show that the phonons of interest are those at the $Z$ point in the hexagonal Brillouin zone of the non-magnetic crystal space group $R-3m$. We use \textit{ab initio} DFT results and compute the Wannierized tight-binding Hamiltonian. From these eigenstates, we compute the first-order response of the axion angle $\partial_{\lambda} \theta$ using the gauge-invariant four curvature.
\end{abstract}

\maketitle

\noindent

\section{Introduction}

The axion angle describes an isotropic contribution to the magnetoelectric coupling in three-dimensional insulators. MnBi$_2$Te$_4$ (MBT) is an $\mathbb{Z}_2$-odd antiferromagnetic topological insulator (AFM-TI) whose axion angle is quantized by the presence of inversion $\mathcal{P}$ and time-reversal times half-translation $\Theta = \mathcal{T}S_{1/2}$ symmetries \cite{mong2010}. Since the quantization is protected by inversion, this puts MBT in the classification of an ``axion insulator" in addition to being a topological insulator. 

MBT forms septuple layers Te-Bi-Te-Mn-Te-Bi-Te, with Mn ions adopting a spin $S = 5/2$ of a $2+$ valence with a large magnetic moment of $\approx 5\ \mu\text{B}$. Below a critical temperature $T_N = 24 \ K$, Mn$^{ 2+}$ moments order ferromagnetically in the ab plane but antiferromagnetically along the crystallographic
c-axis in an A-type antiferromagnetic order \cite{Yan2019}. At the surface, time-reversal symmetry is broken by the AFM ordering, resulting in gapped surface Dirac modes, which leads to a half-quantized surface anomalous Hall conductivity.

In insulating materials that break both spatial inversion and time-reversal symmetries, a magnetic field $B$ can induce an electric polarization $P$ and an electric field $E$ can induce a magnetization $M$. This coupling is described by a linear magnetoelectric coupling tensor $\alpha_{ij} = (\partial P_i / \partial B_j)_{\mathbf{E}=0} = (\partial M_i/\partial E_j)_{\mathbf{B}=0}$ \cite{Malashevich_2010}. The Chern-Simons axion (CSA) coupling makes an isotropic contribution, $\alpha^{\text{CS}}$, to the magnetoelectric coupling proportional to the dimensionless quantity $\theta$,
\begin{equation}
    \alpha^{\text{CS}}_{ij} = \frac{\theta e^2}{2 \pi h} \delta_{ij}.
\end{equation}
The quantity $\theta$, the Chern-Simons axion angle, is determined by the integral of a Chern-Simons 3-form \cite{chern1974characteristic, Malashevich_2010, mong2010antiferromagnetic, essin2010orbital},
\begin{equation}
\label{eq:CS3form}
 \theta = -\frac{1}{4\pi} \int_{\text{BZ}} d^3k \,
\epsilon^{\mu\nu\sigma} \mathrm{Tr} \left[
    \mathcal{A}_\mu \partial_\nu \mathcal{A}_\sigma - \frac{2i}{3} \mathcal{A}_\mu \mathcal{A}_\nu \mathcal{A}_\sigma
\right],
\end{equation}

The electrodynamics in a medium are unaffected by a spatially and temporally uniform $\theta_{\text{CS}}$, being instead only sensitive to its spatial gradients, 
producing local anomalous Hall conductivity: $\mathbf{J}(\mathbf{r}, t) \propto \mathbf{E} \times \nabla \theta_{\text{CS}}(\mathbf{r}, t),$ 
or to its time derivative, producing a local chiral magnetic effect: $\mathbf{J}(\mathbf{r}, t) \propto \mathbf{B} \frac{\partial \theta_{\text{CS}}(\mathbf{r}, t)}{\partial t}.$ 
These effects will be activated insofar as some parameters in the crystal Hamiltonian are spatially inhomogeneous or time-dependent, and their magnitudes will be controlled by the 
first derivatives.

Another motivation connects with recent explorations of the possibility that the axion field, usually treated as a static background field depending on the ground-state crystal Hamiltonian, 
might be promoted to a dynamical field in some contexts [14–17]. It is unlikely that the axion field would have its own independent dynamics and quasiparticles; instead, $\theta_{\text{CS}}$ 
likely varies with preexisting (e.g., phonon or charge-density-wave) degrees of freedom of the solid. Still, such situations could be quite interesting, especially in topological materials where $\theta_{\text{CS}}$ is comparable to or equal to $\pi$.


There is reason to think that the computation of \emph{derivatives} of $\theta_{\text{CS}}$ may actually be more straightforward than the computation of $\theta_{\text{CS}}$ itself. After all, $\theta_{\text{CS}}$ is a 3D analog of the 1D electric polarization $\mathbf{P}$, and historically first derivatives 
of $\mathbf{P}$, such as dynamical charges and dielectric constants, were computed well before the modern theory of polarization. This can be understood because 
$\mathbf{P}$ itself is only well-defined modulo a quantum and is expressed as an integral of a gauge-dependent Berry connection, whereas its derivative 
$\partial_\lambda \mathbf{P}$ does not suffer from either problem.

Similarly, $\partial_\lambda \theta_{\text{CS}}$ can be expressed as a 3D $k$-space integral of the gauge-invariant 4-curvature:
\begin{equation}
F^{(4)} = \frac{1}{16\pi} \epsilon^{ijkl} \text{Tr}(\Omega_{ij} \Omega_{kl}),
\end{equation}
in $(k_x, k_y, k_z, \lambda)$-space, where $\Omega_{ij}$ and $\Omega_{kl}$ are 2D non-Abelian Berry curvatures such as $\Omega_{k_x k_y}$ or $\Omega_{k_x \lambda}$.

The goal of this project is to develop and code practical methods for computing the linear response of $\theta_{\text{CS}}$ to perturbations such as lattice displacements, 
external fields, strain, etc. The implementation will be tested by comparison to finite-difference calculations of the full $\theta_{\text{CS}}$. The main focus will be on 
performing this in the first-principles context using density-functional perturbation theory. The $\lambda$-perturbation will be included using methods from [18].

\section{Axion-active phonons}

To modulate the axion angle, we must seek phonon modes that break the underlying $\theta$-quantizing symmetries. 

\bibliography{ref}

\end{document}